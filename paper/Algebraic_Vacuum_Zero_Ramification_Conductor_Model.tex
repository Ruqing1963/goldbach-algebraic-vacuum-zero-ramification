\documentclass[11pt,a4paper]{article}

% ─── Packages ──────────────────────────────────────────────────────────────────
\usepackage[utf8]{inputenc}
\usepackage[T1]{fontenc}
\usepackage{amsmath,amssymb,amsthm,mathrsfs}
\usepackage{geometry}
\geometry{margin=1in}
\usepackage{xcolor}
\usepackage[breaklinks=true,colorlinks=true,linkcolor=blue!60!black,citecolor=blue!60!black,urlcolor=blue!50!black]{hyperref}
\usepackage{url}
\usepackage{booktabs}
\usepackage{array}
\usepackage{graphicx}
\usepackage{float}

% ─── Theorem environments ──────────────────────────────────────────────────────
\newtheorem{theorem}{Theorem}[section]
\newtheorem{lemma}[theorem]{Lemma}
\newtheorem{proposition}[theorem]{Proposition}
\newtheorem{corollary}[theorem]{Corollary}
\theoremstyle{definition}
\newtheorem{definition}[theorem]{Definition}
\newtheorem{conjecture}[theorem]{Conjecture}
\newtheorem{remark}[theorem]{Remark}
\newtheorem{observation}[theorem]{Observation}

% ─── Operators ─────────────────────────────────────────────────────────────────
\DeclareMathOperator{\ord}{ord}
\DeclareMathOperator{\GSp}{GSp}
\DeclareMathOperator{\Jac}{Jac}
\DeclareMathOperator{\rad}{rad}

% ═══════════════════════════════════════════════════════════════════════════════
\title{The Algebraic Vacuum:\\
Zero-Ramification Conductor Model\\
for the Goldbach Conjecture at $N = 2^k$}
\author{Ruqing Chen\\[4pt]
\textit{GUT Geoservice Inc., Montr\'{e}al, QC, Canada}\\[2pt]
\texttt{ruqing@hotmail.com}}
\date{February 2026}

\begin{document}
\maketitle

% ═══════════════════════════════════════════════════════════════════════════════
\begin{abstract}
When the even integer $N$ is a power of $2$, the Goldbach--Frey curve
$C_{N,p} : y^2 = x(x^2 - p^2)(x^2 - q^2)$ enters a \emph{zero-ramification regime}:
the static conduit factor $\rad_{\mathrm{odd}}(N/2) = 1$, so that the conductor is
governed entirely by the boundary summands $p$ and $q = N - p$.  We exploit this
\emph{algebraic vacuum} to isolate the pure effect of radical rigidity on the
conductor distribution.

Scanning $N = 2^k$ for $k = 7, \ldots, 14$, we establish three structural phenomena:
(1)~\emph{Ground state locking}: the minimum Chen's ratio over Goldbach pairs
satisfies $\rho_{\min} \to 2$ from above, locked by $\rad(p) = p$ for primes;
(2)~\emph{Bandwidth rigidity}: the ratio of composite to Goldbach bandwidth is
consistently $\geq 2.3$ across all tested $k$, demonstrating that the clustering
of Goldbach pairs is a structural invariant, not a statistical artifact;
(3)~\emph{Conductor floor}: Goldbach pairs are uniformly bounded below by
$\rho > 2$, while composite decompositions can reach $\rho = 0$.

These results provide the sharpest available experimental evidence that the
conductor geometry of the Goldbach family is fundamentally constrained by the
algebraic rigidity of prime numbers.
\end{abstract}


% ═══════════════════════════════════════════════════════════════════════════════
\section{Introduction}

The conductor rigidity programme for the Goldbach conjecture, initiated
in~\cite{Chen2026GM} and extended in~\cite{Chen2026GMII}, studies the family of
genus~2 curves
\begin{equation}\label{eq:curve}
  C_{N,p} : y^2 = x(x^2 - p^2)(x^2 - q^2), \qquad q = N - p,
\end{equation}
associated to decompositions of an even integer $N > 4$.  The roots of the
sextic polynomial---$0, \pm p, \pm q$---encode the additive constraint $p + q = N$
into the geometry of a hyperelliptic curve; the bad reduction of its Jacobian
at various primes then reflects the arithmetic obstructions to both $p$ and $q$
being prime.
The discriminant factors as~\cite{Chen2026GM}
\begin{equation}\label{eq:disc}
  \Delta(f) = 2^{12}\, p^6\, q^6\, (M - p)^4\, M^4,
  \qquad M = N/2,
\end{equation}
with the parameter-independent factor $M^4$ constituting the \emph{static conduit}.
In~\cite{Chen2026GMII}, we introduced \emph{Chen's ratio}
\begin{equation}\label{eq:rho}
  \rho(N, p) = \frac{\log \mathcal{N}_{\mathrm{proxy}}}{\log N},
  \qquad
  \mathcal{N}_{\mathrm{proxy}} =
    \bigl(\rad_{\mathrm{odd}}(p)\cdot\rad_{\mathrm{odd}}(q)\cdot
          \rad_{\mathrm{odd}}(M)^2\bigr)^2,
\end{equation}
and demonstrated that Goldbach pairs occupy a narrow $\rho$-band (the ``stability
band'') while composite decompositions spread over a range roughly three times wider.

In the present paper, we specialise to $N = 2^k$, where a remarkable simplification
occurs: $\rad_{\mathrm{odd}}(M) = \rad_{\mathrm{odd}}(2^{k-1}) = 1$, so the static
conduit contributes \emph{nothing} to the conductor.  We call this the
\emph{algebraic vacuum}.  In this regime, every feature of the conductor distribution
arises purely from the boundary summands, making the zero-ramification case the
ideal laboratory for studying the interplay between primality and conductor
geometry.\footnote{Scripts and data:
\url{https://github.com/Ruqing1963/goldbach-algebraic-vacuum-zero-ramification}.}


% ═══════════════════════════════════════════════════════════════════════════════
\section{The Zero-Ramification Regime}

\subsection{Definition and basic properties}

\begin{definition}[Zero-ramification regime]\label{def:zero}
  We say $N$ is in the \emph{zero-ramification regime} if $N = 2^k$ for some
  integer $k \geq 3$.  In this regime, $M = 2^{k-1}$ has no odd prime factors,
  so $\rad_{\mathrm{odd}}(M) = 1$.
\end{definition}

\begin{proposition}[Conductor in the algebraic vacuum]\label{prop:vacuum}
  In the zero-ramification regime, the conductor proxy~\eqref{eq:rho} reduces to
  \begin{equation}\label{eq:vacuum}
    \mathcal{N}_{\mathrm{proxy}}(2^k, p) =
      \bigl(\rad_{\mathrm{odd}}(p) \cdot \rad_{\mathrm{odd}}(q)\bigr)^2,
      \qquad q = 2^k - p.
  \end{equation}
  In particular:
  \begin{enumerate}
  \item The curve $C_{N,p}$ has good reduction at every odd prime $r \nmid p\,q$.
  \item The conductor depends only on the prime factorisation of the boundary
    summands $p$ and $q$; the static conduit plays no role.
  \item For a Goldbach pair $(p, q)$ with both entries prime:
    $\mathcal{N}_{\mathrm{proxy}} = (p \cdot q)^2$ and
    $\rho = 2\log(pq)/(k\log 2)$.
  \end{enumerate}
\end{proposition}

\begin{proof}
  Immediate from $\rad_{\mathrm{odd}}(M) = 1$ and $\rad_{\mathrm{odd}}(r) = r$
  for any odd prime $r$.
\end{proof}

\begin{remark}[Geometric interpretation]\label{rem:geom}
  In the language of arithmetic geometry: at $N = 2^k$, the N\'eron model of
  $\Jac(C_{N,p})$ has good reduction at every odd prime not dividing $p\,q$.
  The only sources of bad reduction are the boundary primes themselves.
  This is the algebraic analogue of a \emph{vacuum state}---all background
  ramification has been switched off, leaving only the ``particle content''
  of the summands.
\end{remark}


\subsection{Three classes of decompositions}

Every decomposition $2^k = p + q$ with $1 < p \leq q$ falls into one of three classes:

\begin{definition}[Decomposition taxonomy]\label{def:taxonomy}
  \mbox{}
  \begin{enumerate}
  \item \textbf{Goldbach (P--P):} Both $p$ and $q$ are prime.
    Then $\rad_{\mathrm{odd}}(p) = p$, $\rad_{\mathrm{odd}}(q) = q$,
    and $\rho = 2\log(pq)/(k\log 2)$.
  \item \textbf{Mixed (P--C):} Exactly one of $p, q$ is prime.
    The prime summand contributes its full value to the radical; the composite
    summand contributes a (potentially much) smaller radical.
  \item \textbf{Composite (C--C):} Both $p$ and $q$ are composite.
    Both radicals can collapse.  In the extreme case $p = q = 2^{k-1}$,
    $\rad_{\mathrm{odd}}(p) = \rad_{\mathrm{odd}}(q) = 1$ and $\rho = 0$.
  \end{enumerate}
\end{definition}


% ═══════════════════════════════════════════════════════════════════════════════
\section{Ground State Locking}\label{sec:ground}

\begin{definition}[Ground state]\label{def:ground}
  For a given $N = 2^k$, the \emph{ground state} is the Goldbach pair $(p_0, q_0)$
  that minimises $\rho$ among all Goldbach pairs.  Since
  $\rho = 2\log(p_0 \cdot q_0)/(k\log 2)$ and $q_0 = 2^k - p_0 > 2^{k-1}$,
  the ground state is achieved by the \emph{smallest} prime $p_0$ such that
  $2^k - p_0$ is also prime.
\end{definition}

\begin{proposition}[Ground state bounds]\label{prop:ground}
  Let $p_0$ be the smallest Goldbach prime for $N = 2^k$.  Then:
  \begin{enumerate}
  \item \textbf{Lower bound:}
    $\rho_{\min} = 2\log(p_0 \cdot q_0)/(k\log 2) > 2$, since
    $p_0 \cdot q_0 = p_0(2^k - p_0) > 2^k = N$ for all $p_0 \geq 3$.
  \item \textbf{Asymptotic:}
    $\rho_{\min} = 2 + 2\log p_0/(k\log 2) + O(p_0 / 2^k)$.
    By the Hardy--Littlewood prediction, $p_0$ grows at most polylogarithmically
    in $N$, so $\rho_{\min} \to 2$ from above as $k \to \infty$.
  \item \textbf{Rigidity:} The bound $\rho > 2$ is an algebraic consequence
    of $p_0 \cdot q_0 > N$, which holds because both factors are prime and
    $q_0 > N/2$.  No composite decomposition is subject to this constraint:
    $(2^{k-1}, 2^{k-1})$ achieves $\rho = 0$.
  \end{enumerate}
\end{proposition}

\begin{proof}
  (i)~Since $p_0 \geq 3$ and $q_0 = 2^k - p_0 \geq 2^k - 3 > 2^{k-1}$ for
  $k \geq 3$, we have $p_0 \cdot q_0 > 3 \cdot 2^{k-1} > 2^k = N$, so
  $\log(p_0 q_0) > k\log 2$ and $\rho > 2$.

  (ii)~Write $q_0 = 2^k(1 - p_0/2^k)$, so
  $\log(p_0 q_0) = \log p_0 + k\log 2 + \log(1 - p_0/2^k)$.
  Since $p_0 \ll 2^k$ for large $k$ (by the Hardy--Littlewood heuristic,
  $p_0$ grows at most polylogarithmically), we have
  $\log(1 - p_0/2^k) = -p_0/2^k + O(p_0^2/2^{2k}) = O(p_0/2^k)$,
  yielding $\log(p_0 q_0) = k\log 2 + \log p_0 + O(p_0/2^k)$.

  (iii)~For the composite pair $(2^{k-1}, 2^{k-1})$, $\rad_{\mathrm{odd}} = 1$
  for both summands, giving $\mathcal{N}_{\mathrm{proxy}} = 1$ and $\rho = 0$.
\end{proof}

Table~\ref{tab:ground} lists the ground states for $k = 7, \ldots, 14$.

\begin{table}[H]
\centering
\begin{tabular}{rrrrrrr}
\toprule
$N$ & $k$ & $p_0$ & $q_0$ & $p_0 \cdot q_0$ & $\rho_{\min}$ \\
\midrule
$128$   & 7  & 19  & 109   & $2\,071$       & 3.148 \\
$256$   & 8  & 5   & 251   & $1\,255$       & 2.573 \\
$512$   & 9  & 3   & 509   & $1\,527$       & 2.350 \\
$1024$  & 10 & 3   & 1021  & $3\,063$       & 2.316 \\
$2048$  & 11 & 19  & 2029  & $38\,551$      & 2.770 \\
$4096$  & 12 & 3   & 4093  & $12\,279$      & 2.264 \\
$8192$  & 13 & 13  & 8179  & $106\,327$     & 2.569 \\
$16384$ & 14 & 3   & 16381 & $49\,143$      & 2.226 \\
\bottomrule
\end{tabular}
\caption{Ground state Goldbach pairs for $N = 2^k$.  When $2^k - 3$ is prime
(as for $k = 9, 10, 12, 14$), the ground state is $(3, 2^k - 3)$ and $\rho_{\min}$
is near $2 + 2\log 3/(k\log 2)$.  The overall trend is $\rho_{\min} \to 2$.}
\label{tab:ground}
\end{table}


% ═══════════════════════════════════════════════════════════════════════════════
\section{Bandwidth Rigidity}\label{sec:bandwidth}

\subsection{The bandwidth ratio}

\begin{definition}[Bandwidth]\label{def:bw}
  For a given $N$ and decomposition class $\mathcal{C} \in \{\text{Goldbach, Mixed,
  Composite}\}$, define
  \[
    \mathrm{BW}_{\mathcal{C}}(N) = \rho_{\max}^{\mathcal{C}} -
      \rho_{\min}^{\mathcal{C}},
  \]
  the width of the $\rho$-band occupied by decompositions of class $\mathcal{C}$.
  The \emph{bandwidth ratio} is $R(N) = \mathrm{BW}_{\mathrm{Comp}} /
  \mathrm{BW}_{\mathrm{GB}}$.
\end{definition}

\begin{proposition}[Bandwidth bounds in the algebraic vacuum]\label{prop:bw}
  For $N = 2^k$ with $k$ sufficiently large:
  \begin{enumerate}
  \item $\mathrm{BW}_{\mathrm{GB}} \leq 2$, since $\rho_{\min} > 2$
    (Proposition~\ref{prop:ground}) and $\rho_{\max} < 4$
    (Proposition~\ref{prop:vacuum}).
  \item $\mathrm{BW}_{\mathrm{Comp}} \to 4$ as $k \to \infty$,
    since composite decompositions can achieve $\rho \approx 0$
    (via smooth numbers) and $\rho \approx 4$ (via near-prime composites
    close to $N/2$).
  \item Consequently, $R(N) \geq 2$ for all sufficiently large $k$.
  \end{enumerate}
\end{proposition}

\subsection{Computational verification}

Table~\ref{tab:bw} presents the bandwidth data across the $2^k$ series.

\begin{table}[H]
\centering
\begin{tabular}{rrrrrrrr}
\toprule
$N$ & $k$ & \#GB & $\rho_{\min}$ & $\langle\rho\rangle$ & $\rho_{\max}$ &
  $\mathrm{BW_{GB}}$ & $R = \mathrm{BW_C / BW_{GB}}$ \\
\midrule
$128$   & 7  & 3   & 3.148 & 3.292 & 3.428 & 0.280 & $10.56$ \\
$256$   & 8  & 8   & 2.573 & 3.203 & 3.490 & 0.917 & $3.38$ \\
$512$   & 9  & 11  & 2.350 & 3.310 & 3.555 & 1.204 & $2.66$ \\
$1024$  & 10 & 22  & 2.316 & 3.291 & 3.600 & 1.284 & $2.56$ \\
$2048$  & 11 & 25  & 2.770 & 3.444 & 3.636 & 0.866 & $3.86$ \\
$4096$  & 12 & 53  & 2.264 & 3.424 & 3.667 & 1.403 & $2.43$ \\
$8192$  & 13 & 76  & 2.569 & 3.532 & 3.692 & 1.123 & $3.07$ \\
$16384$ & 14 & 151 & 2.226 & 3.551 & 3.714 & 1.488 & $2.34$ \\
\bottomrule
\end{tabular}
\caption{Bandwidth evolution across $N = 2^k$.  The ratio $R$ is consistently
$\geq 2.3$, confirming that Goldbach bandwidth is structurally narrower than
composite bandwidth by a factor that does not shrink with $k$.}
\label{tab:bw}
\end{table}

\begin{observation}[Structural stability of $R$]\label{obs:R}
  The bandwidth ratio $R(2^k)$ fluctuates between $2.3$ and $10.6$ but never
  drops below $2.3$ for $k \leq 14$.  The lower bound $R \geq 2$ from
  Proposition~\ref{prop:bw} is thus comfortably satisfied.  The fluctuations
  in $R$ are driven primarily by $\rho_{\min}$: when $2^k - 3$ is prime,
  the ground state drops close to $\rho = 2$, widening the Goldbach bandwidth
  and decreasing $R$.  When $2^k - 3$ is composite, $p_0$ is larger, the
  Goldbach band is narrower, and $R$ increases.
\end{observation}


% ═══════════════════════════════════════════════════════════════════════════════
\section{The Conductor Landscape at $N = 1024$}\label{sec:1024}

We present a detailed case study at $N = 2^{10} = 1024$, which has 22 Goldbach
pairs, 127 mixed decompositions, and 361 composite-only decompositions.

\begin{table}[H]
\centering
\begin{tabular}{llrrr}
\toprule
Type & $(p, q)$ & $\rad_{\mathrm{odd}}(p)$ & $\rad_{\mathrm{odd}}(q)$ & $\rho$ \\
\midrule
Goldbach (ground)  & $(3,\; 1021)$   & 3      & 1021   & $\mathbf{2.316}$ \\
Goldbach           & $(5,\; 1019)$   & 5      & 1019   & $2.463$ \\
Goldbach           & $(11,\; 1013)$  & 11     & 1013   & $2.689$ \\
Goldbach (ceiling) & $(509,\; 515)$  & 509    & 515    & $3.600$ \\
\midrule
Composite (smooth) & $(512,\; 512)$  & 1      & 1      & $0.000$ \\
Composite (smooth) & $(256,\; 768)$  & 1      & 3      & $0.317$ \\
Composite (smooth) & $(128,\; 896)$  & 1      & 7      & $0.562$ \\
Composite (high)   & $(507,\; 517)$  & 507    & 517    & $3.600$ \\
\bottomrule
\end{tabular}
\caption{Conductor metrics at $N = 1024$.  The Goldbach band is
$\rho \in [2.316, 3.600]$ (width $1.284$); the composite range is
$\rho \in [0.000, 3.600]$ (width $3.600$).  The ``ground state'' $(3, 1021)$
has $\rho = 2.316$, while the composite ``floor'' $(512, 512)$ reaches $\rho = 0$.}
\label{tab:1024}
\end{table}

Figure~\ref{fig:landscape} shows the complete decomposition landscape.

\begin{figure}[H]
  \centering
  \includegraphics[width=\textwidth]{figure1_N1024.pdf}
  \caption{\textbf{Left}: Chen's ratio for all 510 decompositions of $N = 1024$.
  Goldbach pairs (blue, $n = 22$) are bounded below at $\rho = 2.316$ and cluster
  in a narrow band.  Composite decompositions (orange, $n = 361$) spread from
  $\rho = 0$ to $\rho = 3.6$.  The ground state $(3, 1021)$ is annotated.
  \textbf{Right}: Density histogram confirming the clustering.
  Mean gap $\Delta\rho = 0.650$.}
  \label{fig:landscape}
\end{figure}

Table~\ref{tab:1024} and Figure~\ref{fig:landscape} together illustrate the full
conductor anatomy of the algebraic vacuum.  The composite pair $(512, 512) = (2^9, 2^9)$
achieves the absolute minimum $\rho = 0$: both summands are pure powers of~2,
their odd radicals are~$1$, and the conductor proxy collapses to~$1$.  This is the
``total radical collapse''---the deepest point in the conductor landscape, accessible
only to smooth numbers.  The next-lowest composites, $(256, 768)$ and $(128, 896)$,
retain a single odd prime factor ($3$ or $7$ respectively), lifting $\rho$ to
$0.317$ and $0.562$.  By contrast, the Goldbach ground state $(3, 1021)$ sits at
$\rho = 2.316$---more than $2.3$ units above the composite floor---because
$\rad_{\mathrm{odd}}(3) = 3$ and $\rad_{\mathrm{odd}}(1021) = 1021$ admit no collapse.
The gap between $\rho = 0$ (composite floor) and $\rho > 2$ (Goldbach floor) is
the conductor-geometric manifestation of the fundamental distinction between primes
and composites.

\begin{remark}[The conductor floor]\label{rem:floor}
  The most striking feature of Figure~\ref{fig:landscape} is the sharp lower
  boundary of the Goldbach points at $\rho \approx 2.3$.  No Goldbach pair can
  penetrate below $\rho = 2$ (Proposition~\ref{prop:ground}), while composite
  decompositions freely descend to $\rho = 0$.  This asymmetry is the defining
  signature of the algebraic vacuum: in the absence of background ramification,
  the radical rigidity of primes ($\rad(p) = p$) imposes an absolute conductor
  floor that composites do not obey.
\end{remark}


% ═══════════════════════════════════════════════════════════════════════════════
\section{Implications for the Goldbach Conjecture}\label{sec:implications}

\subsection{The radical rigidity mechanism}

The zero-ramification model isolates the fundamental mechanism underlying the
conductor geometry of the Goldbach family:

\begin{enumerate}
\item \textbf{Primes are rigid.}  For a prime $p$, $\rad(p) = p$.  This means
  the conductor proxy $\mathcal{N}_{\mathrm{proxy}} = (pq)^2$ is completely
  determined by $p$ and $q$ without any radical collapse.

\item \textbf{Composites are compressible.}  For a composite $p$ with
  $\omega(p) \geq 2$, the radical satisfies $\rad(p) < p$, and for smooth
  numbers $\rad(p) \ll p$.  This allows the conductor to collapse.

\item \textbf{The gap is structural.}  The conductor floor at $\rho > 2$ for
  Goldbach pairs, versus $\rho = 0$ for composites, is not a statistical
  phenomenon but an algebraic one: it follows from the identity $\rad(p) = p$
  for primes, which is a \emph{tautology} of the definition of primality.
\end{enumerate}

\subsection{Connection to the static conduit}

For general $N$ (not a power of 2), the static conduit factor
$\rad_{\mathrm{odd}}(M)^2$ in~\eqref{eq:rho} provides an additional floor that
raises the entire $\rho$-distribution.  The zero-ramification regime $N = 2^k$
thus represents the \emph{worst case} for the Goldbach conductor floor: with
$\rad_{\mathrm{odd}}(M) = 1$, the floor is at its lowest ($\rho \to 2$), and
any Goldbach pair that exists must survive purely on the strength of its boundary
primes.  The fact that Goldbach pairs remain rigidly clustered even in this worst
case is the strongest available evidence for the structural robustness of the
Goldbach conductor geometry.

\subsection{The open gap}

\begin{remark}[What is and what is not established]\label{rem:honest}
  This paper establishes:
  \begin{itemize}
  \item The algebraic vacuum model and its simplification of the conductor
    proxy (Proposition~\ref{prop:vacuum}).
  \item The ground state floor $\rho > 2$ for Goldbach pairs, proved
    unconditionally (Proposition~\ref{prop:ground}).
  \item The bandwidth rigidity $R \geq 2.3$ across $k = 7, \ldots, 14$,
    verified computationally (Table~\ref{tab:bw}).
  \end{itemize}
  What it does \emph{not} establish is the \emph{existence} of Goldbach pairs
  for every $N = 2^k$.  The conductor floor tells us where Goldbach pairs must
  sit if they exist, but not that they exist.  The existence question remains
  equivalent to the Goldbach conjecture, and would require effective
  equidistribution results for the $\GSp(4)$ family as discussed
  in~\cite{Chen2026GMII}.
\end{remark}


% ═══════════════════════════════════════════════════════════════════════════════
\section{Comparison with the General Case}\label{sec:comparison}

Table~\ref{tab:compare} compares the zero-ramification regime with the general
case studied in~\cite{Chen2026GMII}.

\begin{table}[H]
\centering
\renewcommand{\arraystretch}{1.2}
\begin{tabular}{>{\raggedright}p{4cm} p{4.5cm} p{4.5cm}}
\toprule
& \textbf{General $N$}~\cite{Chen2026GMII} & \textbf{$N = 2^k$} (this paper) \\
\midrule
Static conduit $\rad_{\mathrm{odd}}(M)$
  & $\geq 1$ (usually $\gg 1$)
  & $= 1$ (vanishes) \\
Conductor proxy
  & $(\rad_{\mathrm{odd}}(p)\!\cdot\!\rad_{\mathrm{odd}}(q)\!\cdot\!\rad_{\mathrm{odd}}(M)^2)^2$
  & $(\rad_{\mathrm{odd}}(p)\!\cdot\!\rad_{\mathrm{odd}}(q))^2$ \\
Goldbach $\rho$-band
  & Centred at $\rho \approx 6.8$
  & Centred at $\rho \approx 3.3$--$3.5$ \\
Goldbach floor
  & $\rho > 2 + 2\log\rad_{\mathrm{odd}}(M)/\log N$
  & $\rho > 2$ (pure boundary) \\
Bandwidth ratio $R$
  & $\approx 3$
  & $\geq 2.3$ (worst case) \\
Singular series
  & $\prod_{r|M} (r\!-\!1)/(r\!-\!2)$
  & $= 1$ (no correction) \\
\bottomrule
\end{tabular}
\caption{Zero-ramification regime versus the general case.  The $N = 2^k$ case
is the ``hardest'' for the conductor floor (lowest $\rho_{\min}$, no singular
series boost), yet the bandwidth rigidity persists.}
\label{tab:compare}
\end{table}


% ═══════════════════════════════════════════════════════════════════════════════
\section{Conclusion}

The zero-ramification regime at $N = 2^k$ provides the cleanest window into the
conductor geometry of the Goldbach family.  By eliminating the static conduit
entirely, it isolates the fundamental mechanism: the radical rigidity of prime
numbers forces Goldbach pairs into a narrow conductor band, while composite
decompositions spread freely over a range at least $2.3$ times wider.

The ground state locking at $\rho > 2$ is proved unconditionally and constitutes
an algebraic invariant: it depends only on $\rad(p) = p$ for primes, which is
the defining property of primality itself.  This suggests that the conductor
geometry of the Goldbach conjecture is not an artifact of analytic approximations,
but reflects a deep algebraic structure that persists even in the most
``transparent'' arithmetic environments.


% ═══════════════════════════════════════════════════════════════════════════════
\section*{Acknowledgments}

The zero-ramification scan was performed using \texttt{goldbach\_rigid\_scan.py},
available at
\url{https://github.com/Ruqing1963/goldbach-algebraic-vacuum-zero-ramification}.
This work builds on the conductor rigidity framework developed in~\cite{Chen2026GM}
(repository: \url{https://github.com/Ruqing1963/goldbach-mirror-conductor-rigidity})
and~\cite{Chen2026GMII}
(repository: \url{https://github.com/Ruqing1963/goldbach-mirror-II-geometric-foundations}).


% ═══════════════════════════════════════════════════════════════════════════════
\begin{thebibliography}{10}

\bibitem{Chen2026CI}
R.~Chen, \emph{Conductor incompressibility for Frey curves associated to prime gaps},
Zenodo, 2026.
\url{https://zenodo.org/records/18682375}

\bibitem{Chen2026W}
R.~Chen, \emph{Weil restriction rigidity and prime gaps via genus 2 hyperelliptic
Jacobians}, Zenodo, 2026.
\url{https://zenodo.org/records/18683194}

\bibitem{Chen2026GM}
R.~Chen, \emph{The Goldbach mirror: conductor rigidity and the static conduit in
$\GSp(4)$}, Zenodo, 2026.
\url{https://zenodo.org/records/18684892}

\bibitem{Chen2026GMII}
R.~Chen, \emph{The Goldbach mirror~{II}: geometric foundations of conductor
rigidity and the static conduit in $\GSp(4)$}, Zenodo, 2026.
\url{https://zenodo.org/records/18719056}

\bibitem{HL1923}
G.\,H.~Hardy and J.\,E.~Littlewood, \emph{Some problems of `Partitio Numerorum';
III: On the expression of a number as a sum of primes}, Acta Math.\
\textbf{44} (1923), 1--70.

\bibitem{Ogg1967}
A.\,P.~Ogg, \emph{Elliptic curves and wild ramification},
Amer.\ J.\ Math.\ \textbf{89} (1967), 1--21.

\end{thebibliography}

\end{document}
